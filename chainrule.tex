\documentclass[a4paper]{article}

\usepackage{indentfirst}
\usepackage{tikz}
\usepackage{float}
\usepackage[stable, hang, flushmargin, bottom]{footmisc}
\usepackage{enumitem}
\usepackage{amsmath}
\usepackage{amssymb}
\usepackage[utf8]{inputenc}
\usepackage[margin=1.0in]{geometry}

\title{One-Dimensional Chain Rule: Demonstration}
\author{Aurelien {\sc Lourot}}
\date{2014-12-30}

\begin{document}
    \sloppy

    \maketitle

    This article demonstrates the {\it chain rule}, a formula for computing the
    derivative of the composition of two functions, first used by Leibniz 
    (around 1700). It states that

    \bigskip
    given two real functions $f$ and $g$ ($\mathbb{R} \rightarrow \mathbb{R}$)
    and $x_0 \in \mathbb{R}$, 

    \bigskip
    if $g$ is differentiable at $x_0$ and $f$ is differentiable at $g(x_0)$, 

    \bigskip
    then
    $$
        (f \circ g)'(x_0) = (f' \circ g)(x_0) \cdot g'(x_0)
    $$

    \section{Alternate notations}

    The {\it chain rule} is often written in the following more accessible way:
    \begin{equation}
        \label{simplenotation}
        {d \over dx} (f(g(x))) = f'(g(x)) \cdot g'(x)
    \end{equation}
    where the free variable $x_0$ has been replaced by $x$.

    \bigskip
    The word {\it chain} suddenly springs into focus when using {\it Leibniz's 
    notation}: if one defines $u = g(x)$ and $y = f(u)$, 
    \begin{itemize}
      {
        \item $f'$ represents the variation of $y$ due to the variation of $u$,
              i.e. $dy \over du$;
      }
      {
        \item $g'$ represents the variation of $u$ due to the variation of $x$,
              i.e. $du \over dx$.
      }
    \end{itemize}

    \bigskip
    Substituting these terms in \eqref{simplenotation} results in
    $$
        {dy \over dx} = {dy \over du} \cdot {du \over dx}
    $$

    \section{Demonstration}

    The definition of $g$ being differentiable at $x_0$ is
    $$
        \lim_{a \to 0}{{g(x_0+a) - g(x_0)} \over a}
    $$
    exists. Let's name it $b_0$.
    \begin{equation}
        \label{gdifferentiable}
        \Leftrightarrow \exists h : \mathbb{R} \rightarrow \mathbb{R},
        \forall a \in \mathbb{R}^*, {{g(x_0+a) - g(x_0)} \over a} = b_0 + h(a)
    \end{equation}
    with
    $$
        \lim_{a \to 0}{h(a)} = 0
        \,\,\,\,^(\footnote
        {
          Note that there might also be one or several non null values of $a$,
          even ``far'' from 0, for which $h(a)$ is exactly null.
        }
    $$

    \bigskip
    \begin{tabular}{rcl}
        \eqref{gdifferentiable} & $\Leftrightarrow$ & 
        $g(x_0+a) = g(x_0) + a \cdot b_0 + a \cdot h(a)$ \\
        & $\Rightarrow$ & 
        $f(g(x_0+a)) = f(g(x_0) + a \cdot b_0 + a \cdot h(a))$
    \end{tabular}

    \bigskip
    \bigskip
    Given $k(a) = a \cdot b_0 + a \cdot h(a)$, we have
    \begin{equation}
        \label{fwithk}
        f(g(x_0+a)) = f(g(x_0) + k(a))
    \end{equation}

    \bigskip
    \centerline{---------------------------}
    \bigskip

    Thanks to $f$ being differentiable at $g(x_0)$, we know that {\bf for $k(a)$
    non null}
    $$
        \lim_{k(a) \to 0}{{f(g(x_0)+k(a)) - f(g(x_0))} \over k(a)}
    $$
    exists. Let's name it $c_0$.
    \begin{equation}
        \label{fdifferentiable}
        \Leftrightarrow \exists l : \mathbb{R} \rightarrow \mathbb{R},
        {{f(g(x_0)+k(a)) - f(g(x_0))} \over k(a)} = c_0 + l(k(a))
    \end{equation}
    when $k(a) \not= 0$ with
    \begin{equation}
        \label{liml}
        \lim_{b \to 0}{l(b)} = 0
        \,\,\,\,^(\footnote
        {
          $\eqref{liml} \Rightarrow \lim_{a \to 0}{l(k(a))} = 0$
        }
    \end{equation}

    \bigskip
    When $k(a) \not= 0$,
    \begin{tabular}{rcl}
        \eqref{fdifferentiable} & $\Leftrightarrow$ & 
        $f(g(x_0)+k(a)) = f(g(x_0)) + k(a) \cdot c_0 + k(a) \cdot l(k(a))$
    \end{tabular}

    \bigskip
    \centerline{---------------------------}
    \bigskip

    Let's define $m : \mathbb{R} \rightarrow \mathbb{R}$ as
    $$
        m(c) = 
        \begin{cases}
          l(c), & c \neq 0; \\
          0, & c = 0.
        \end{cases}
    $$

    \bigskip
    We now have, for any value of $k(a)$, 
    $f(g(x_0)+k(a)) = f(g(x_0)) + k(a) \cdot c_0 + k(a) \cdot m(k(a))$

    \bigskip
    Substituting this result into \eqref{fwithk} gives
    $$
        f(g(x_0+a)) = f(g(x_0)) + k(a) \cdot c_0 + k(a) \cdot m(k(a))
    $$
    $$
        \Leftrightarrow f(g(x_0+a)) - f(g(x_0)) = 
        [a \cdot b_0 + a \cdot h(a)] \cdot c_0 
        + [a \cdot b_0 + a \cdot h(a)] \cdot m(k(a))
    $$
    $$
        \Leftrightarrow {{f(g(x_0+a)) - f(g(x_0))} \over a} = 
        b_0 \cdot c_0 + h(a) \cdot c_0 
        + [b_0 + h(a)] \cdot m(k(a))
    $$
    $$
        \Rightarrow \lim_{a \to 0}{{f(g(x_0+a)) - f(g(x_0))} \over a} = 
        b_0 \cdot c_0
    $$

    \bigskip
    Substituting $b_0$ and $c_0$ by their respective definitions gives
    \begin{equation}
        \label{threelim}
        \lim_{a \to 0}{{f(g(x_0+a)) - f(g(x_0))} \over a} = 
        \left(\lim_{a \to 0}{{g(x_0+a) - g(x_0)} \over a}\right) \cdot 
        \left(\lim_{k(a) \to 0}{{f(g(x_0)+k(a)) - f(g(x_0))} \over k(a)}\right)
    \end{equation}

    \bigskip
    Note that all three limit arguments ($a$, $a$ and $k(a)$) are independant
    free variables, so that \eqref{threelim} can be re-written as
    $$
        \lim_{a \to 0}{{f(g(x_0+a)) - f(g(x_0))} \over a} = 
        \left(\lim_{b \to 0}{{g(x_0+b) - g(x_0)} \over b}\right) \cdot 
        \left(\lim_{c \to 0}{{f(g(x_0)+c) - f(g(x_0))} \over c}\right)
    $$

    $$
        \Leftrightarrow
        \boxed
        {
          (f \circ g)'(x_0) = g'(x_0) \cdot f'(g(x_0))
        }
    $$
    \begin{flushright}
    $\blacksquare$
    \end{flushright}

\end{document}
